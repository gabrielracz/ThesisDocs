\documentclass[a4paper, 11pt, titlepage]{article}
% \usepackage{geometry}
% \usepackage{longtable}
\usepackage{tabularx}
\usepackage{setspace}
\usepackage[backend=biber]{biblatex}
\addbibresource{references.bib}

% \nocite{*}

\newcommand{\vmiddle}[1]{\begin{tabular}{l} #1 \end{tabular}}
\renewcommand\tabularxcolumn[1]{m{#1}}% for vertical centering text in X column


\title{COMP4906\\
Honours Thesis Pre-Proposal\\
Real-Time 3D Volumetric Fire Simulation For Games}
\author{Gabriel Racz}

\begin{document}
 \maketitle

\section{Motivation}
In recent years, rapid improvements in consumer graphics hardware has allowed game
developers to create more detailed worlds through new rendering and simulation
techniques. AAA studios have begun leveraging GPU compute to simulate game
systems such as volumetric light scattering \cite{RDR2}, and interactive wind
simulations \cite{GOW}. One area that hasn't seen rapid adoption by industry is
the simulation and rendering of interactive volumetric fire. Although
interactive fire has seen an increase of representation in new titles such as Read
Dead Redemption 2, The Legend of Zelda: Breath of the Wild, and Far Cry 5, these
techniques largely rely on simplistic spreading interactions and either sprite-based
particle rendering.

2018's God of War utilized a basic volumetric wind fluid simulation to allow to
interact with vegetation, cloth, and hair \cite{GOW}. I believe
there is potential to push this idea further and incorporate a more
detailed fluid simulation that handles fire, boundary conditions with game
objects, and volumetric lighting.

\section{Objective}
The objective of this thesis is to research methods that when combined will
result in a efficient GPU-based volumetric fire and fluid simulation for
the use in games on current generation hardware. The intended use case would be
medium-scale simulated burning in the environment around the player with a focus
on trees and vegetation.
 This will require working
within memory and performance constraints such that it can be run in
real-time, while still budgeting performance for other gameplay systems. To
achieve pleasant visuals and high performance, a balance of physically based
models and artistic parameters will be necessary. 

As the field of fire simulation spans many decades, thorough literature review
will be conducted in order to find areas for improvement and novelty.
Firstly, I will review and implement some of the standard techniques used
in past literature for 
real-time fluid dynamics such as the famous Stable Fluids \cite{stam}, as well
as other widely referenced works on the simulation and rendering of fire
\cite{nguyen} \cite{gems}. I intend to develop a rendering and simulation
engine using the Vulkan graphics API to facilitate GPGPU implementation of the
various algorithms involved in the thesis. I opt for a custom engine to allow
for complete control of the graphics hardware in order to achieve my goal of
simulating and rendering the fluids at 60 fps or higher. The overarching
constraint of the thesis will be developing a technique that can realistically
be used in modern games, meaning the total frame budget for both the simulation
and rendering passes should at minimum be below 16ms.

\pagebreak
Below are some areas I have identified as potentially providing interesting extensions to
existing methods.

\begin{itemize}
    \item Adaptive/hierarchical grid simulation domain. In the majority of past research
    uniform and static voxel grids are used to simulate the fluid volume.
    Optimizing the cubic memory and runtime requirements of uniform grids could offer
    large performance benefits in terms of physics integration speed,
    level of detail over sparce game worlds, physical interactions
    with other game objects, and volumetric lighting. I will investigate ``Multigrid
    integration'' \cite{DCGRID}, dynamic octrees/bounding volume hierarchies,
    simulation LODs, and selectively budgeting higher resolution simulation/rendering to areas of
    higher detail.1
    \item Simulation grid uspcaling utilizing noise and other non-physical techniques
    to fill in gaps between grid cells. This would offer a method to induce
    turbulence and other chaotic phenomena that are observed in real fire while
    saving simulation integration cost. Curl Noise \cite{curl}, Wavelet
    Turbulence \cite{wavelet}, or other procedural noise methods should be
    investigated. Experiment with combining grid and particle-based approaches
    (Eulerian and Lagrangian fluids) to reduce grid resolution needed for fine detail.
    \item Rendering methods for volumetric lighting and interactions between
    flames, smoke, and game geometry. I will research techniques for rendering the flames
    themselves, including level set shells \cite{nguyen}, ray
    marching/tracing, or other procedural texturing solutions. Investigate
    methods for representing volumetric light and shadow within the smoke
    volume using the existing fluid grid.
    \item Interaction with game geometry. Using game geometry to represent fuel
    and allowing for the dynamic spreading of flames through a scene. Handle
    boundary conditions between the fluids and game geometry. Allow for the
    player to influence the fluid volume through their actions.
\end{itemize}

\section{Tentative Schedule}

\begin{tabularx}{\textwidth}{|X|X|}
\hline
\textbf{Period} & \textbf{Milestones} \\
\hline
September 9 - September 22 & 
\multicolumn{1}{X|}{Initial literature review. Research Stable Fluids \cite{stam} and other well
established methods for real-time fluid simulation suitable for games.} \\
\hline
September 23 - October 6 & \multicolumn{1}{X|}{
Begin developing the Vulkan engine with the goal of
reimplementing the techniques within \cite{stam} and \cite{nguyen} using a GPGPU
pipeline.} \\
\hline
October 7 - October 20 & \multicolumn{1}{X|}{
Finish working prototype of 3D fluid simulation, measure
performance and identify initial bottlenecks worth further research.
Begin implementing raster-based pipeline for mesh
geometry (and continue whenever time is available).} \\
\hline
October 21 - November 3 & \multicolumn{1}{X|}{
Add combustion  to the fluid model (heat transfer, reactants, smoke). In parallel continue researching modern methods
for rendering and simulating flames. Identify which physical models are
necessary for realistic visuals.
}\\
\hline
November 4 - November 17 & \multicolumn{1}{X|}{
Use gathered experience to inform areas of future study. Research methods for
adaptive grids. Experiment with the addition of noise and/or upscaling.
Submit rough draft of full proposal.
}\\
\hline
November 18 - December 1 & \multicolumn{1}{X|}{
Use feedback from proposal rough draft to further direct experiments and
literature review. Commit to thesis content.
}\\
\hline
December 2 - December 15 & \multicolumn{1}{X|}{
Finalize and submit thesis proposal. Time permitting, continue more focused
reading of papers and engine improvements.
}\\
\hline
\end{tabularx}

\pagebreak
\printbibliography
\end{document}